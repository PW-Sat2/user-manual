\telecommand{Perform Payload Commissioning Experiment}

\tccategory{PD}
\subsection{General description}
\texttt{PerformPayloadCommissioningExperiment} can be used to request performing a payload 
commissioning experiment at first given opportunity. 

\paragraph{Known issues} \mbox{} \\
\None

\paragraph{Side effects} \mbox{} \\
This command overrides previously requested experiments that has not been started yet. 

\paragraph{Usage limitations} \mbox{} \\
This command does not ensure that payload commissioning experiment will be started 
by the time that response is sent back to the operator. It only requests performing 
an experiment, which will be done at earliest opportunity.

\paragraph{Recommendations for operation team} \mbox{} \\
The experiment will overwrite selected output file if it already exists therefore ensure 
that selected file does not already exists or is no longer needed or some important data
may be lost.

\subsection{Definition}
\TelecommandApid{PerformPayloadCommissioningExperiment}
\TelecommandDeclaration{PerformPayloadCommissioningExperiment}

\begin{tcarglist}
    \tcarg{correlation\_id}{8}{Correlation ID}
    \tcarg{file\_name}{up to 30 * 8}{Path to file that should contain experiments results}
\end{tcarglist}

\subsection{Responses}
\ResponseApid{ExperimentSuccessFrame}
\paragraph{ExperimentSuccessFrame}

Correlated response frame \texttt{ExperimentSuccessFrame} confirming 
that \radfet experiment performance has been requested.

\paragraph{ExperimentErrorFrame}
Correlated response frame \texttt{ExperimentErrorFrame} representing an error. 
Error code indicates failure reason:

\begin{tabular}{r | l}
    Error code & Value \\
    \hline
    \texttt{0x00}   & Success \\
    \texttt{0x01}   & Malformed (too short) telecommand \\
    \texttt{0x02}   & Another experiment may already be running \\
\end{tabular}

\subsection{Examples}
Request performing payload commissioning experiment with results saved to 
to '/payload' file with communication correlation id set to 20.

\exampleCall{PerformPayloadCommissioningExperiment(
    correlation_id=20,
    file_name='/payload'
)}

