\telecommand{Abort Experiment}

\tccategory{PD}

\subsection{General description}
\texttt{AbortExperiment} telecommand aborts currently running experiment.\\
Success frame does not indicate that abort was successful.\\
If no experiment is running, the command is ignored.\\
The abort \textbf{is not} immediate, it is executed during next loop of \nameref{obc:proc:Experiment loop}.

\paragraph{Known issues}
\begin{itemize}
	\item Some experiments [\todo{List of such experiments?}] are written in one loop iteration so they can't be aborted.
	\item Operation is not immediate. State change is applied during next Mission Loop iteration and next Experiment Loop iteration.
\end{itemize}

\paragraph{Side effects}
\begin{itemize}
	\item Experiment is not fully executed.
	\item Result of Experiment is unpredictable (generated files, executed actions)
	\item Subsystems started during experiment execution may be left in undefined state (enabled, not fully commissioned, ...).
\end{itemize}

\paragraph{Usage limitations}\mbox{}\\ 
\None

\paragraph{Recommendations for operation team}
\begin{itemize}
	\item Check Experiment's manual for possible effects of aborting.
	\item Plan additional steps in next session to retrieve state of satellite.
	\item Nominal state of subsystems can be reverted by \nameref{obc:proc:Power cycle}.
\end{itemize}

\subsection{Definition}
\TelecommandApid{AbortExperiment}
\ctor{AbortExperiment}

Parameters: 

\begin{tabular}{r | l | l}
	Argument                    & Size [bits] & Description \\
	\hline
	\texttt{correlation\_id}    & 8 		  &	Correlation ID \\
\end{tabular}

\subsection{Responses}

\paragraph{ExperimentSuccessFrame}
\ResponseApid{ExperimentSuccessFrame}
Correlated response frame \texttt{ExperimentSuccessFrame} confirms that telecommand was received and Abort of Experiment will be performed. It DOES NOT mean that it has be succesfully executed.
No extra payload is carried in this response frame.

\paragraph{ExperimentErrorFrame}
\ResponseApid{ExperimentErrorFrame}
Correlated response frame \texttt{ExperimentErrorFrame} represents error. Error code indicates failure reason:

\begin{tabular}{r | l}
	Error code & Value \\
	\hline
	\texttt{0x01} & Malformed (too short) telecommand \\
	
\end{tabular}

\subsection{Example usage}
Abort experiment, correlation ID = 11
\binaryForm{AbortExperiment(11)}
