\telecommand{Enter Idle State}

\tccategory{D}

\subsection{General description}
\texttt{EnterIdleState} telecommand allows switching COMM to Idle State mode for given time.

In this mode when Transmitter is idle it will continuously transmit Idle Sequence (0x7E) to help the ground receiver to lock.
When Mission Time advances to time set in telecommand, COMM restart or Power Cycle the Idle State is automatically disabled.

\warn{Leaving satellite in this state too long will discharge battery and may cause transmitter overheating and failure.}
\\\\
Success frame indicates that transmitter entered Idle State.

\paragraph{Known issues}
\begin{itemize}
	\item Maximum Idle Time duration is 255 seconds (4 m 15 s)
	\item All response error codes return the same value.
	\item Time of disabling Idle State is based on Mission Time, not some specialized timer. That may introduce small variations in time.
\end{itemize}

\paragraph{Side effects}
\begin{itemize}
	\item \warn{Above nominal power consumption and transmitter heating.}
\end{itemize}

\paragraph{Usage limitations}\mbox{}\\ 
This command should not be used when battery level is low.

\paragraph{Recommendations for operation team}
\begin{itemize}
	\item \textbf{Do not} leave Idle State active beyond communication window.
	\item Monitor power usage and transmitter temperature during Idle State.
	\item If possible use repeated \nameref{tc:Send Beacon} or \nameref{tc:Ping} telecommand instead.
\end{itemize}

\paragraph{Power budget} \mbox{} \\
Idle state duration vs consumed energy chart is shown below. Think twice if you need to use this telecommand.

\begin{pycode}
from pylab import *

TX_PEAK_PWR_mW = 3000
def set_idle_wh_consumption(duration):
    return (duration / 3600) * TX_PEAK_PWR_mW

clf()
figure(figsize=(7, 3.5))
rcParams['font.family'] = 'serif'
duration_seconds = linspace(1, 255, 255)

plot(duration_seconds, set_idle_wh_consumption(duration_seconds), 'r', label="dd")
grid(True)
title("Idle state duration vs consumed energy")
xlabel("Idle state duration [seconds]")
ylabel("Consumed energy [mWh]")
ylim(0, 250)
xlim(1, 255)
fill_between(duration_seconds, 100, 250, facecolor='red', alpha=0.2)
text(60, 175, "DANGER ZONE!", color='red')

savefig("enteridlestate.pdf", bbox_inches="tight")

print(r"\begin{center}")
print(r"\includegraphics[width=0.9\textwidth]{enteridlestate.pdf}")
print(r"\end{center}")
\end{pycode}

\subsection{Definition}
\TelecommandApid{EnterIdleState}
\TelecommandDeclaration{EnterIdleState}

Parameters: 

\begin{tabular}{r | l | l}
	Argument                    & Size [bits] & Description \\
	\hline
	\texttt{correlation\_id}    & 8 		  &	Correlation ID \\
	\texttt{duration}			& 8 		  & Time in seconds how long COMM should be kept in Idle State
\end{tabular}

\subsection{Responses}

\paragraph{CommSuccessFrame}
\ResponseApid{CommSuccessFrame}
Correlated response frame \texttt{CommSuccessFrame} confirms that telecommand was received COMM entered Idle State.

\paragraph{CommErrorFrame}
\ResponseApid{CommErrorFrame}
Correlated response frame \texttt{CommErrorFrame} represents error. Error code indicates failure reason:

\begin{tabular}{r | l}
	Error code & Value \\
	\hline
	\texttt{0x01} & Malformed (too short) telecommand \\ 
	 			  & Failed to read current time for setting timeout \\
				  & Failed to enter Idle State by COMM
	
\end{tabular}

\subsection{Example usage}
Enter Idle State for 2 minute 15 seconds (135 seconds), correlation ID = 11
\exampleCall{EnterIdleState(11, 135)}
