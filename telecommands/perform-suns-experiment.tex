\telecommand{Perform SunS Experiment}

\tccategory{PD}

\subsection{General description}
\texttt{Perform SunS Experiment} telecommand turns The SunS Experiment. Collected data are saved in on-board flash memory in two separate filenames. For more details refer to The SunS Experiment description.

\paragraph{Known issues} \mbox{} \\
\None

\paragraph{Side effects} \mbox{} \\
\begin{itemize}
    \item SENS_5V and SUNS_3V3 lines are switched on by the experiment.
\end{itemize}


\paragraph{Usage limitations} \mbox{} \\
\begin{itemize}
    \item Do not use in case of PLD board or reference Sun sensor (SENS_5V power line) malfunctions, or The SunS fault (SUNS_3V3 power line).
    \item Do not use when battery level is low.
\end{itemize}

\paragraph{Recommendations for operation team}
\begin{itemize}
    \item Calculate power consumption by the experiment itself and data transmission before sending telecommand using Telecommands Analyzer.
    \item Running the experiment take into account Guidelines for SunS Experiment.
    \item Schedule the time of the experiment so it's finished before next communication session.
    \item Download experiment's secondary file at least once (at least a part of it).
\end{itemize}

\subsection{Definition}
\TelecommandApid{PerformSunSExperiment}
\TelecommandDeclaration{PerformSunSExperiment}

Parameters:

\begin{tcargslist}   
    \tcarg{correlation\_id}{8}{int}{Correlation ID}
    \tcarg{gain}{8}{int}{Gain parameter of ALS sensors (BH1730FVC)}
    \tcarg{itime}{8}{int}{ITime parameter of ALS sensors (BH1730FVC)}
    \tcarg{samples_count}{8}{int}{Number of samples in one iteration of the experiment}
    \tcarg{short_delay}{8}{datetime.timedelta}{Delay between consecutive samples (one-second resolution)}
    \tcarg{session_count}{8}{int}{Number of iterations}
    \tcarg{long_delay}{8}{datetime.timedelta}{Delay between consecutive iterations with decasecond resolution}
    \tcarg{filename}{max. 160}{string}{Base \texttt{filename}. Experiment creates also the second file with name \texttt{filename_sec}}
\end{tcargslist}


\subsection{Responses}

\paragraph{ExperimentSuccessFrame}
\ResponseApid{ExperimentSuccessFrame}

Correlated frame of type \texttt{ExperimentSuccessFrame} indicates that 

\paragraph{ExperimentErrorFrame}
\ResponseApid{ExperimentErrorFrame}

Correlated frame of type \texttt{ExperimentErrorFrame} indicates error during telecommand processing. Error code value carries failure reason

\begin{tabular}{r | l}
    Error code & Value \\
    \hline
    \texttt{1} & Malformed (too short) telecommand \\   
    \texttt{2} & Error while requesting experiment start \\ 
\end{tabular}

\subsection{Example usage}
Perform SunS Experiment, with gain \texttt{1} (\texttt{gain=0x01}), itime \texttt{10} (\texttt{itime=0x10}) and use correlation ID \texttt{11} (\texttt{correlation\_id=11}). There are \texttt{10} sampling sessions (\texttt{session_counts=5}) with \texttt{5} samples (\texttt{samples_count=5}). Between each sampling session there is \texttt{500} s of delay (\texttt{long_delay=50}), whereas between consecutive samples there is \texttt{1} s of delay (\texttt{short_delay=1})  Created experiment files are: \texttt{"suns_1"} and \texttt{"suns_1_sec"}. 

\exampleCall{PerformSunSExperiment(11, 0x01, 0x10, 5, 1, 10, 50, "suns_1")}