\telecommand{Get Persistent State}

\tccategory{S}

\subsection{General description}
\texttt{GetPersistentState} retrieves current content of satellite's \nameref{obc:proc:Persistent State}. 

\paragraph{Known issues} \mbox{} \\
\None

\paragraph{Side effects} \mbox{} \\
\None

\paragraph{Usage limitations} \mbox{} \\
\None

\paragraph{Recommendations for operation team} \mbox{} \\
When storing retrieved persistent state data, note the exact date and time of retrival. 

\subsection{Definition}
\TelecommandApid{GetPersistentState}
\TelecommandDeclaration{GetPersistentState}

No parameters are required for this telecommand.

\subsection{Responses}
\responseFrame{PersistentStateFrame}
\ResponseApid{PersistentStateFrame}

Response frame is of type \texttt{PersistentStateFrame} will contain entire persistent state. There's no correlation id for this telecommand.

Response frame will contain all variables as defined in \nameref{obc:proc:Persistent State}. All variables will be sent in first frame. Since \texttt{MessageState::message} can have up to 200 bytes, if it doesn't fit in the first frame, second frame will be sent with the rest of the message.

\subsection{Example usage}
Retrieve persistent state:
\exampleCall{GetPersistentState()}