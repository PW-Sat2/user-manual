\telecommand{Set Time}

\tccategory{PD}

\subsection{General description}
\texttt{SetTime} allows changing mission time to arbitrary values overriding time correction mechanism. Essentially it provides a way to ``jump through time'' both to future and past.

After invoking this telecommand new mission time should be visible in telemetry.

Details of Time tracking are described in \procref{Mission Time}.

\paragraph{Known issues} \mbox{}\\
\None

\paragraph{Side effects}
\begin{itemize}
	\item Mission time is changed.
	\item If mission time is moved beyond T=40 days, emergency sail opening will begin.
\end{itemize}

\paragraph{Usage limitations}
\begin{itemize}
	\item If sail opening procedure has already started, moving time backwards \textbf{WILL NOT} prevent sail from opening
	\item Moving time back inside LEOP timeframe \textbf{WILL} start LEOP experiment again after OBC restart.
\end{itemize}

\paragraph{Recommendations for operation team}
\begin{itemize}
	\item Use only to correct clock skews that slows mission time considerably (more than 25\%)
	\item In case of major clock skews analyze time source measurements as it may indicating MCU clock/MCU RTC/external RTC failure. See \tcref{Set Time Correction Config} for more information on how to disable RTCs.
\end{itemize}


\subsection{Definition}
\TelecommandApid{SetTime}
\TelecommandDeclaration{SetTime}

Parameters: 

\begin{tcarglist}
	\tcarg{correlation\_id}{8}{Correlation ID}
	\tcarg{new\_time}{32}{New mission time in seconds}
\end{tcarglist}


\subsection{Responses}

\paragraph{TimeSetSuccessFrame}
\ResponseApid{TimeSetSuccessFrame}
Correlated response frame \texttt{TimeSetSuccessFrame} represents success and carries additional status of the set time operation:

\begin{tabular}{r | l}
	Status & Value \\
	\hline
	\texttt{0} & Operation performed correctly \\	
	\texttt{1} & Failed to retrieve time from external RTC. Last known value has been used \\
\end{tabular}

\paragraph{TimeSetErrorFrame}
\ResponseApid{TimeSetErrorFrame}
Correlated response frame \texttt{TimeSetErrorFrame} represents error. Error code indicates failure reason:

\begin{tabular}{r | l}
	Error code & Value \\
	\hline
	\texttt{-2} & Unable to set time \\	
	\texttt{-3} & Cannot set time state \\	
	\texttt{-4} & Unable to acquire time synchronization lock \\
	\texttt{-5} & Cannot get time state \\
\end{tabular}



\subsection{Example usage}
Set mission time to 20 days 13 hours 56 minutes 4 seconds (50164 seconds in total), correlation ID = 11
\exampleCall{SetTime(11, datetime.timedelta(days=20, hours=13, minutes=56, seconds=4))}
