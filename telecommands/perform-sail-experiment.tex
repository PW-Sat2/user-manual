\telecommand{Perform Sail Experiment}

\tccategory{D}

\subsection{General description}
\texttt{Perform Sail Experiment} starts sail experiment. \todo{Describe sail experiment}

\danger{\textbf{This command ends the mission.}}

Although similar to \nameref{tc:Open Sail} this command schedules full SAIL experiment which gathers data and takes photos. SAIL opening procedure is the same as in \nameref{tc:Open Sail} (see \nameref{obc:proc:Sail Deployment Process}).

\paragraph{Known issues} \mbox{} \\ 
\begin{itemize}
	\item Stopping the sail experiment does not stop sail deployment process.
\end{itemize}

\paragraph{Side effects}
\begin{itemize}
	\item SAIL experiment is started.
	\item A lot of frames are sent during experiment.
	\item Increased power usage
	\item End of mission
	\item Very low probability of successful communication
	\item A lot of photo files which causes longer boot and increased memory usage
\end{itemize}

\paragraph{Usage limitations}
\begin{itemize}
	\item Use only when satellite is in good condition
\end{itemize}

\paragraph{Recommendations for operation team}
\begin{itemize}
	\item Use only when mission is going well.
	\item If satellite condition indicate possible problems use \nameref{tc:Open Sail} to open sail without performing experiment.
	\item Ensure proper power levels before performing experiment.
	\item Prepare to receive a lot of frames during next 4 minutes.
	\item Schedule data and photo download for next sessions (if possible)
	\item When experiment is running, the \texttt{OBC_Experiments_Code} telemetry field should be set to 7 (SAIL experiment), \texttt{OBC_Experiments_LastIterationStatus} should be set to 2 (LoopImmediately) or 3 (WaitForNextCycle). 
\end{itemize}


\subsection{Definition}
\TelecommandApid{PerformSailExperiment}
\TelecommandDeclaration{PerformSailExperiment}

Parameters: 

\begin{tcarglist}
	\tcarg{correlation\_id}{8}{Correlation ID}
\end{tcarglist}


\subsection{Responses}

\paragraph{ExperimentSuccessFrame}
\ResponseApid{ExperimentSuccessFrame}
Correlated response frame \texttt{ExperimentSuccessFrame} represents success. 

\paragraph{ExperimentErrorFrame}
\ResponseApid{ExperimentErrorFrame}
Correlated response frame \texttt{ExperimentErrorFrame} represents error. Error code indicates failure reason:

\begin{tabular}{r | l}
	Error code & Value \\
	\hline
	\texttt{1} & Malformed (too short) telecommand \\	
	\texttt{2} & Error while requesting experiment start \\	
\end{tabular}


\subsection{Example usage}
Perform sail experiment, correlation ID = 11
\exampleCall{PerformSailExperiment(11)}

