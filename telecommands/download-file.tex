\telecommand{Download File}

\tccategory{PD}

\subsection{General description}
\texttt{DownloadFileTelecommand} can be used to download single or multiple chunks of a file.

\paragraph{Known issues} \mbox{} \\
\None

\paragraph{Side effects} \mbox{} \\
\begin{itemize}
    \item Beacon frames might be delayed because of large amonunt od response frames with the chunks. 
\end{itemize}

\paragraph{Usage limitations} \mbox{} \\
\begin{itemize}
    \item Do not use when battery level is low. Check it always before sending the 
    \item Do not request more than 30 chunks at once or you can power-cycle the satellite.
\end{itemize}

\paragraph{Recommendations for operation team} \mbox{} \\
\begin{itemize}
    \item Analyze energy consumption by the telecommand, do not ruin power budget!
    \item Check power amplifier temperatures before, during (if possible) and after the telecommand usage. Do not overheat COMM module.
    \item Be patient - send the telecommand once and wait for response frames at least a few seconds (up to 10 - 15 s TBD) before repeating the telecommand.
    \item It's recommended to request not more than 10 - 15 chunks in one telecommand. Uplink has some non-zero PER, moreover some lag between telecommand reception by
    the PW-Sat2 and sending reponse frames with files chunks is present. The lag may cause a situation when an operator will send the telecommand again (as one would think that the telecommand was lost/not received by the spacecraft), even if the first telecommand was actually accepted. It may lead to unintentional power-cycle if more than 30 chunks will be queued by the OBC.
\end{itemize}
This telecommand can be used to download images done with camera or to download a raw telemetry.

\subsection{Definition}
\TelecommandApid{DownloadFile}
\TelecommandDeclaration{DownloadFile}

Parameters: 

\begin{tcarglist}
	\tcarg{correlation\_id}{8}{Correlation ID}
	\tcarg{path}{variable}{Path to the file on the satellite's filesystem}
	\tcarg{seqs}{variable}{32-bit LE sequence numbers of chunks that will be send}
\end{tcarglist}

\subsection{Responses}

\paragraph{FileSendSuccessFrame}
\ResponseApid{FileSendSuccessFrame}

This telecommand always responds with \texttt{FileSendSuccessFrame}.

The first byte indicates success or error code.

\begin{tabular}{r | l}
	Code & Description \\ \hline
	0 		& Success \\
	1 		& File not found \\
	2 		& Malformed request \\
	3 		& Path is invalid \\
	4 		& Sequence is too big
\end{tabular}

For \texttt{Success} the response follows with \texttt{requested\_chunks} amount of the corresponding file segment bytes.

For \texttt{File not found} and \texttt{Sequence is too big} error codes the response follows with received file path without the terminating zero.

For \texttt{Malformed request} and \texttt{Path is invalid} error codes the response doesn't contain any additional data.

\subsection{Example usage}
Download 1, 10 and 65536 chunks (\texttt{seqs=1, 10, 65536}) from picture.jpg file (\texttt{path=/picture.jpg}) \texttt{correlation\_id=11}.

\exampleCall{DownloadFile(11, "/picture.jpg", [1, 10, 65536])}

A response should consist of three \texttt{CorrelatedDownlinkFrame} that contain three chunks from the picture.jpg file.