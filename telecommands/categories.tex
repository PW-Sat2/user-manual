\section{Categories}

Each telecommand has assigned one out of three defined categories (safe, potentially dangerous, dangerous), based on objective rating given according to descriptions provided in this chapter.

\subsection{Safe telecommand}
\tccategory{S}
\vspace{1em}
\begin{itemize}
    \item Telecommand has little impact on the spacecraft,
    \item it does not change state, mode or behaviour of the spacecraft,
    \item it does not turn on or off any module of the satellite,
    \item when used with compliance with \textbf{general rules and constraints} it should not be considered as a source of any fault.
\end{itemize}

\subsection{Potentially dangerous telecommand}
\tccategory{PD}
\vspace{1em}
\begin{itemize}
    \item Telecommand has significant, but reversible impact on the spacecraft,
    \item it does change state, mode or behaviour of the spacecraft, but in normal/expected conditions, this change is not destructive,
    \item it can turn on or off any module of the satellite,
    \item when used with compliance with \textbf{general rules and constraints} it must not cause any damages or changes that may affect communication and control of the spacecraft by the operation team, or introduced by the telecommand changes may be effectively reversed by the spacecraft itself.
\end{itemize}


\subsection{Dangerous telecommand}
\tccategory{D}
\vspace{1em}
\begin{itemize}
    \item Telecommand has significant, hardly reversible or irreversible impact on the spacecraft,
    \item it does change state, mode or behaviour of the spacecraft in any way,
    \item it can turn on or off any module of the satellite, including subsystems designed to run limited number of times,
    \item even when used with compliance with \textbf{general rules and constraints} it may cause damages or changes that may affect communication and control of the spacecraft by the operation team in unpredicted way.
\end{itemize}
