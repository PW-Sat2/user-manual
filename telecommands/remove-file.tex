\telecommand{Remove file}

\tccategory{PD}

\subsection{General description}
\texttt{Remove File} telecommand can be used to remove single file from \OBC flash memory. 

\paragraph{Known issues} \mbox{} \\
\begin{itemize}
    \item It is possible to remove current/archival telemetry storage files by using this command, which
    results of loosing the information what was happening recently. 
\end{itemize}

\paragraph{Side effects} \mbox{} \\
Requested file is immediately removed.

\paragraph{Recommendations for operation team}
\begin{itemize}
    \item Take special care to ensure that file selected for removal is either fully downloaded and
    backed up or no longer needed.
    \item \textbf{Do not} leave no longer needed files in memory as the increasing number of files has negative effect 
    on the \todo{List Files} telecommand.
\end{itemize}

\paragraph{Usage limitations}\mbox{}\\ 
It is not possible to remove file whose path not including null terminator is longer than 194 bytes.

\subsection{Definition}
\TelecommandApid{RemoveFile}
\TelecommandDeclaration{RemoveFile}

Parameters: 

\begin{tcarglist}
    \tcarg{correlation\_id}{8}{Correlation ID}
    \tcarg{path\_length}{8}{Size of the file path in bytes. This argument is automatically calculated based on the path argument. This value should not  include null terminator.}
    \tcarg{path}{\texttt{path\_length} * 8}{Path to the file that should be removed. }
\end{tcarglist}

\subsection{Responses}
\ResponseApid{FileRemoveSuccessFrame}
\paragraph{FileRemoveSuccessFrame}
Correlated response frame \texttt{FileRemoveSuccessFrame} confirming that requested file has been removed.

\paragraph{FileRemoveErrorFrame}
Correlated response frame \texttt{FileRemoveErrorFrame} representing an error. Error code indicates failure reason:

\begin{tabular}{r | l}
    Error code & Value \\
    \hline
    \texttt{0x00}   & Success \\
    \texttt{0x16}   & Malformed (too short) telecommand \\
                    & Invalid argument \\
    any other & File removal error code \\
\end{tabular}

Both \texttt{FileRemoveSuccessFrame} and \texttt{FileRemoveErrorFrame} should contain path to file that 
has been removed or failed to remove.


\subsection{Example usage}
Remove file "photo.jpg" with correlated Id set to 11.

\exampleCall{RemoveFile(11, 'photo.jpg')}
