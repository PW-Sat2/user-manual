\telecommand{Power cycle}

\tccategory{PD}

\subsection{General description}
\texttt{PowerCycle} performs full satellite power cycle using EPS. Power cycle attempts will performed according to \nameref{obc:proc:Power cycle} procedure. Just before performing power cycle success frame is sent to acknowledge receipt of telecommand. It \textbf{does not} indicate that power cycle was successful.

Figure \ref{fig:tc:powercycle} shows flow of power cycle telecommand. If power cycle was successful processing will be stopped after issuing \texttt{PowerCycle()}.

\begin{figure}[h]	
	\centering
	
	 \begin{sequencediagram}
		\newinst{gs}{Operator}{}
		\newinst[5]{obc}{OBC}{}    
		
		\begin{call}{gs}{PowerCycle}{obc}{PowerFailureFrame}
			\mess{obc}{PowerSuccessFrame}{gs}{}
			
			\begin{callself}{obc}{PowerCycle()}{Failure}
			\end{callself}
			
		\end{call}
		
	\end{sequencediagram}
		
	\caption{Power cycle ordered by telecommand}
	\label{fig:tc:powercycle}
\end{figure}

\paragraph{Known issues}
\begin{itemize} 
	\item Confirmation response frame might not be sent if power cycle will be faster than COMM module
	\item It is not possible to force usage of specific EPS controller to perform power cycle. If control over used EPS controller is required, \nameref{tc:Raw I2C Transfer} should be used.
\end{itemize}

\paragraph{Side effects}
\begin{itemize}
	\item Satellite is power cycled. 
	\item No communication in the next few minutes.
\end{itemize}

\paragraph{Usage limitations}\mbox{}\\ 
\None

\paragraph{Recommendations for operation team}
\begin{itemize}
	\item Plan regular power cycles.
	\item There will be no communication with sattelite during power cycle.
	\item Perform power cycles during long sessions to be able to verify that satellite is back in operational state.
	\item Expected OBC telemetry state after successfull power cycle: incremented \texttt{OBC_Startup_BootCounter} field and reset \texttt{OBC_Uptime} field. 
	\item Only one of the EPS controllers will perform it's power cycle. 
		\begin{itemize} 
			\item If Controller A rebooted, then \texttt{EPS_A_PowerCycleCounter} field will be incremented and \texttt{EPS_A_Uptime} will be reset.
			\item If Controller B rebooted, then \texttt{EPS_B_PowerCycleCounter} field will be incremented and \texttt{EPS_B_Uptime} will be reset.
		\end{itemize}	
	\item Expected IMTQ telemetry state after successfull power cycle: reset \texttt{IMTQ_State_Uptime} field.	
\end{itemize}


\subsection{Definition}
\TelecommandApid{PowerCycleTelecommand}
\TelecommandDeclaration{PowerCycleTelecommand}

Parameters: 

\begin{tcarglist}
	\tcarg{correlation\_id}{8}{Correlation ID}
\end{tcarglist}


\subsection{Responses}

\paragraph{PowerSuccessFrame}
\ResponseApid{PowerSuccessFrame}
Correlated response frame \texttt{PowerSuccessFrame} confirms that telecommand was received and Power Cycle will be performed. It DOES NOT mean that power cycle was successful. No extra payload is carried in this response frame.

\paragraph{PowerErrorFrame}
\ResponseApid{PowerErrorFrame}
Correlated response frame \texttt{PowerErrorFrame} represents error. Error code indicates failure reason:

\begin{tabular}{r | l}
	Error code & Value \\
	\hline
	\texttt{0x01} & Malformed (too short) telecommand \\
	\texttt{0x02} & Power cycle error (\OBC execution was not terminated after issuing power cycle command) \\
	
\end{tabular}

\subsection{Example usage}
Perform power cycle, correlation ID = 11
\exampleCall{PowerCycleTelecommand(11)}

