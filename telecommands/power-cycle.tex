\subsection{Power cycle}
\texttt{PowerCycle} performs full satellite power cycle using EPS. Power cycle attempts will performed according to \nameref{obcproc:powercycle} procedure. Just before performing power cycle success frame is sent to acknowledge receipt of telecommand. It \textbf{does not} indicate that power cycle was successful.

Figure \ref{fig:tc:powercycle} shows flow of power cycle telecommand. If power cycle was successful processing will be stopped after issuing \texttt{PowerCycle()}.

\begin{figure}[h]
	\label{fig:tc:powercycle}
	\centering
	
	 \begin{sequencediagram}
		\newinst{gs}{Operator}{}
		\newinst[5]{obc}{OBC}{}    
		
		\begin{call}{gs}{PowerCycle}{obc}{PowerFailureFrame}
			\mess{obc}{PowerSuccessFrame}{gs}{}
			
			\begin{callself}{obc}{PowerCycle()}{Failure}
			\end{callself}
			
		\end{call}
		
	\end{sequencediagram}
		
	\caption{Power cycle ordered by telecommand}
\end{figure}

\subsubsection{Success response}
Success response frame is of type \texttt{PowerSuccessFrame} and contains no additional payload.

\subsubsection{Error response}
Error frame is of type \texttt{PowerErrorFrame} and contains error code:
\begin{itemize}
	\item Malformed (too short) telecommand - error code 1
	\item Failed to power cycle - error code 2
\end{itemize}

\subsubsection{Definition}
\ctor{PowerCycleTelecommand}

Arguments: 
\begin{description}[labelindent=1cm]
	\item[\texttt{correlation\_id}] - Correlation ID (8-bit number)
\end{description}

\subsubsection{Example usage}