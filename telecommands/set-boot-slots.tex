\section{Set boot slots}

\tccategory{D}

\subsection{General description}
\texttt{SetBootSlots} telecommand allows switching boot slots used for booting OBC. 

There are 6 boot slots storing exact copies of OBC firmware. They are divided into 2 groups: 3 Main slots and 3 Failsafe slots. 
The parameters are bitmasks where 3 slots for each group must be chosen.

Bootloader using TMR boots main program from Main Slots or after 10 failed attempts from Failsafe Slots.
There als also two special cases: Upper Boot Slot, using during development and Safe Mode boot slot used during abnormal conditions.

\paragraph{Known issues}
\begin{itemize}
    \item \danger{It is possible to set the same slots as Main and Failsafe.}
	\item \danger{It is possible to set UpperBoot slot and SafeMode slot as Main or Failsafe.}
	\item Change is not reflected in telemetry as value in telemetry says from which slots program was started, not what is stored in those slots.
\end{itemize}

\paragraph{Side effects}
\begin{itemize}
	\item Main and Failsafe slots are changed.
	\item Next reboot will use new vales of slots.
\end{itemize}

\paragraph{Usage limitations} 
\begin{itemize}
	\item Should be used only during emergency scenarios and during software upload process.
\end{itemize}

\paragraph{Recommendations for operation team} 
\None

\subsection{Definition}
\TelecommandApid{SetBootSlots}
\TelecommandDeclaration{SetBootSlots}

Parameters:

\begin{tcarglist}
	\tcarg{correlation\_id}{8}{Correlation ID}
	\tcarg{primary}{8 bit mask}{Arithmetical OR of 3 values below}
	\tcarg{failsafe}{8 bit mask}{Arithmetical OR of 3 values below}
\end{tcarglist}

Values for \texttt{primary} and \texttt{failsafe} arguments:

\begin{tabular}{r | r | l}
    Value (Bin) & Value (Hex) & Slot \\
    \hline
	0b00000001 & 0x01 & Slot 0 \\
	0b00000010 & 0x02 & Slot 1 \\
	0b00000100 & 0x04 & Slot 2 \\
	0b00001000 & 0x08 & Slot 3 \\
	0b00010000 & 0x10 & Slot 4 \\
	0b00100000 & 0x20 & Slot 5 \\
	0b01000000 & 0x40 & \danger{Safe Mode Slot} \\
	0b10000000 & 0x80 & \danger{Upper Boot Slot} \\
\end{tabular}

\subsection{Responses}

\responseFrame{BootSlotsInfoSuccessFrame}
\ResponseApid{BootSlotsInfoSuccessFrame}

Correlated success response frame is of type \texttt{BootSlotsInfoSuccessFrame} and contains 3 bytes: 

\begin{tabular}{l | l}
    Item & value \\
    \hline
	Success indicator & 0 \\
	Current value of Main slots & values as in above table \\
	Current value of Failsafe slots & values as in above table \\
\end{tabular}

If this frame is sent, change of boot slots was successfully saved in \todo{Boot Settings}.

\responseFrame{BootSlotsInfoErrorFrame}
\ResponseApid{BootSlotsInfoErrorFrame}

Correlated error response frame type type is \texttt{BootSlotsInfoErrorFrame} and contains error code:

\begin{tabular}{r | l}
    Error code & Value \\
    \hline
    0xE0 (224) & Malformed request (i.e. missing parameters) \\
    0xE1 (225) & Invalid Main Bootslots - cannot set persistent state \\
    0xE2 (226) & Invalid Failsafe Bootslots - cannot set persistent state \\
\end{tabular}

\subsection{Example usage}
Set boot slots, respond with correlation ID 12, Set slots 0, 1, 2 (0x07) as Main, 3, 4, 5 (0x38) as Failsafe
\exampleCall{SetBootSlots(12, 0x7, 0x38)}
