\telecommand{Raw I2C Transfer}

\tccategory{D}

\subsection{General description}

\danger{This is "admin mode" command. You can do everything possible on I2C, including bricking satellite.}

\warn{This command requires knowledge about I2C communication and protocol, hardware internals, subsystem operations and data protocols.}

\texttt{RawI2C} performs raw data transfer on selected I2C bus.
Two modes are available:
\begin{itemize}
	\item Direct Write-Read, when no delay between operations are provided.
	\item Write-Delay-Read, for some devices that require special timing during communication. 
		\begin{itemize}
			\item Antenna (both controllers)
			\item COMM (both controllers)
			\item iMTQ
		\end{itemize}
\end{itemize}

\paragraph{Known issues}
\begin{itemize}
	\item Success response frame is sent even if I2C transfer failed. Then error code is in first byte of response.
	\item OBC always reads from device 230 bytes (full frame). Operator should know response data length meaningful for provided command.
\end{itemize}

\paragraph{Side effects}
\begin{itemize}
	\item \warn{This is command allowing direct interaction with hardware. Side effects can't be listed in advance.}
\end{itemize}

\paragraph{Usage limitations}\mbox{}\\ 
\None

\paragraph{Recommendations for operation team}
\begin{itemize}
    \item Don't use it unless you \textbf{really} have to.
    \item Be familiar with I2C addressing and data transfers.
    \item Check documentation about parameters and possible side effects.
	\item Double check you provide correct data and device address.
	\item \warn{Test command on flatsat.}
	\item Check first byte of response - there can be error indicator.
\end{itemize}

\subsection{Definition}
\TelecommandApid{RawI2C}
\TelecommandDeclaration{RawI2C}

Parameters: 

\begin{tcarglist}
	\tcarg{correlation\_id}{8}{Correlation ID}
	\tcarg{busSelect}{8}{Selected bus. (see list below for avaiable values)}
	\tcarg{address}{8}{Device address.}
	\tcarg{delay}{16}{Delay in ms between Write and Read operations.}
	\tcarg{data}{max 190 * 8}{Data to write.}
\end{tcarglist}

Available buses:

\begin{tabular}{r | l}
	Hex value & selected bus \\ \hline
	\texttt{0x00} & System Bus. \\
	\texttt{0x01} & Payload Bus. \\
	\texttt{default} & Error. 
\end{tabular}

\subsection{Responses}

\paragraph{I2CSuccessFrame}
\ResponseApid{I2CSuccessFrame}
Correlated response frame \texttt{I2CSuccessFrame} confirms that telecommand was received and I2C transfer has ended. It does not mean that transfer is successful.

First byte of response is status indicator:

\begin{tabular}{r | l | l}
	Status indicator & Code & Description \\
	\hline
	\texttt{0} & OK & Transfer completed successfully. \\
	\texttt{-1} & Nack & NACK received during transfer. \\
	\texttt{-2} & BusErr & Bus error during transfer (misplaced START/STOP). \\
	\texttt{-3} & ArbLost & Arbitration lost during transfer. \\
	\texttt{-4} & UsageFault & Usage fault. \\
	\texttt{-5} & SwFault & Software fault. \\
	\texttt{-6} & Timeout & Timeout error. \\
	\texttt{-7} & LineLatched & SCL or SDA line is latched low at the end of transfer. \\
	\texttt{-8} & Failure & General I2C error. \\
	\texttt{-9} & LineAlreadyLatched & SCL or SDA line is latched low before transfer. \\
\end{tabular}

Remaining part of response frame is zero-initialized buffer with data read on I2C bus. 

\paragraph{I2CErrorFrame}
\ResponseApid{I2CErrorFrame}
Correlated response frame \texttt{I2CErrorFrame} represents error. Error code indicates failure reason:

\begin{tabular}{r | l}
	Error code & Value \\
	\hline
	\texttt{0x01} & Malformed (too short) telecommand \\
	\texttt{0x02} & Unknown bus selected \\
\end{tabular}

\subsection{Example usage}
Reset COMM transmitter, correlation ID = 11, System bus (0), Address = 97, delay = 100 ms, data = 170 (reset command code)
\exampleCall{RawI2C(11, 0, 97, 100, [170])}
