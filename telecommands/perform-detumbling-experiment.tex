\telecommand{Perform Detumbling Experiment}

\tccategory{D}

\subsection{General description}
\texttt{PerformDetumblingExperiment} telecomand can be used to request performing a \detumbling 
experiment at first given opportunity.

\paragraph{Known issues} \mbox{} \\
\textcolor{red}{\bfseries{Experimental detumbling exhibits unstable behaviour and MUST NOT be turned on!}}

\paragraph{Side effects} \mbox{} \\
This command overrides previously requested experiments that has not been started yet. 

\paragraph{Usage limitations} \mbox{} \\
This command does not ensure that \detumbling experiment will be started by the time that 
response is sent back to the operator. It only requests performing an experiment, which 
will be done at earliest opportunity.

\paragraph{Recommendations for operation team} \mbox{} \\
The experiment will overwrite selected output file if it already exists therefore ensure 
that selected file does not already exists or is no longer needed or some important data
may be lost.

\subsection{Definition}
\TelecommandApid{PerformDetumblingExperiment}
\TelecommandDeclaration{PerformDetumblingExperiment}

Parameters:

\begin{tcarglist}
    \tcarg{correlation\_id}{8}{Correlation ID}
    \tcarg{duration}{8}{Experiment duration in seconds}
    \tcarg{sampling\_interval}{8}{Time delay in seconds between subsequent \detumbling measurements}
\end{tcarglist}

\subsection{Responses}
\ResponseApid{ExperimentSuccessFrame}
\paragraph{ExperimentSuccessFrame}

Correlated response frame \texttt{ExperimentSuccessFrame} confirming 
that \detumbling experiment performance has been requested.

\paragraph{ExperimentErrorFrame}
Correlated response frame \texttt{ExperimentErrorFrame} representing an error. 
Error code indicates failure reason:

\begin{tabular}{r | l}
    Error code & Value \\
    \hline
    \texttt{0x00}   & Success \\
    \texttt{0x01}   & Malformed (too short) telecommand \\
    \texttt{0x02}   & Another experiment may already be running \\
\end{tabular}

\subsection{Examples}
Request performing \detumbling experiment with 4 samples taken once every second with communication correlation id set to 12.

\exampleCall{PerformRadFETExperiment(
    correlation_id=12,
    duration=4,
    sampling_interval=1)}
