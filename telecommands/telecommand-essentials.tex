\subsection{Telecommand essentials}

\textbf{Category: danegorus/possibly dangerous/safe} (just an idea, maybe we should have an icon for each level, and some box around)

\subsubsection{General description}
\begin{itemize}
    \item Short and general description of a telecommand (e.g. \textbf{purpose}, \textbf{common usage}).
    \item It should be followed by \textbf{detailed mechanizm of operation}. It is also acceptable to put a link to more detailed description located in different part of the document.
\end{itemize}

Then, following chapters \textbf{must} be present:
\begin{itemize}
    \item \textbf{"Known issues"} - e.g. 'sail can be opened once only',
    \item \textbf{"Side effects"} - e.g. 'increasing a baud rate, power consumption decreases',
    \item \textbf{"Usage limitations"} - e.g. 'do not use in particular mode', or 'when condition x is below y',
    \item \textbf{"Recommendations for operation team"} - e.g. 'beacons/telemetry are saved a bit slower when files grow, take into account computing next chunks'.
\end{itemize}

\subsubsection{Structure}

\paragraph{Parameters}
    \begin{itemize}
    \item table of parameters, or "no parameters"
    \end{itemize}


\paragraph{Response}
    \textbf{Success Frame definition}
    \begin{itemize}
        \item table with possible responses
        \item binary forms if possible
    \end{itemize}


    \textbf{Error Frame definition}
    \begin{itemize}
        \item table with possible responses
        \item binary forms if possible
    \end{itemize}


\subsubsection{Example usage}

\begin{itemize}
    \item short description of the case
    \item parameters' values with justification/description
    \item Python code
    \item binary form of the telecommand
\end{itemize}
