\telecommand{Perform RadFET Experiment}

\tccategory{PD}

\subsection{General description}
\texttt{PerformRadFETExperiment} telecomand can be used to request performing an \radfet 
experiment at first given opportunity. The experiment demands longer period of bias prior to data aquisition. 
This delay is controlled by a telecommand parameter. After the delay, aqusition of given samples amount is started 
with maximum possible rate (about 25 s per sample). The rate cannot be adjusted.



\paragraph{Known issues} \mbox{} \\
\None

\paragraph{Side effects} \mbox{}
\begin{itemize}
    \item This command overrides previously requested experiments that has not been started yet. 
    \item The experiment turns on SENS_5V power line along with \radfet LCL located on PLD board.
    \item Energy consumption is increased by 0.25 W from 5V_SENSE line (0.36 W from BP).
\end{itemize}


\paragraph{Usage limitations} \mbox{} \\
This command does not ensure that \radfet experiment will be started by the time that 
response is sent back to the operator. It only requests performing an experiment, which 
will be done at earliest opportunity.

\paragraph{Recommendations for operation team} \mbox{}
\begin{itemize}
    \item The experiment will overwrite selected output file if it already exists therefore ensure 
          that selected file does not already exists or is no longer needed or some important data
          may be lost.
    \item Time of experiment execution can be very long - about 3 hours. Watch out energy usage!
    \item Mandatory use Task Analyzer tool to determine energy and storage usage.
    \item Run the experiment always in the same (or at least similar) conditions. The experiment is sensitive
          to temperature variations!

\end{itemize}

\subsection{Definition}
\TelecommandApid{PerformRadFETExperiment}
\TelecommandDeclaration{PerformRadFETExperiment}

Parameters: 

\begin{tcarglist}
    \tcarg{correlation\_id}{8}{Correlation ID}
    \tcarg{delay}{8}{Time biasing delay in decaseconds (tens of seconds) prior to \radfet measurements}
    \tcarg{samples\_count}{8}{Number of measurements to take}
    \tcarg{output\_file\_name}{up to 30 * 8}{Path to file that should contain experiments results}
\end{tcarglist}

\subsection{Responses}
\ResponseApid{ExperimentSuccessFrame}
\paragraph{ExperimentSuccessFrame}

Correlated response frame \texttt{ExperimentSuccessFrame} confirming 
that \radfet experiment performance has been requested.

\paragraph{ExperimentErrorFrame}
Correlated response frame \texttt{ExperimentErrorFrame} representing an error. 
Error code indicates failure reason:

\begin{tabular}{r | l}
    Error code & Value \\
    \hline
    \texttt{0x00}   & Success \\
    \texttt{0x01}   & Malformed (too short) telecommand \\
    \texttt{0x02}   & Another experiment may already be running \\
\end{tabular}

\subsection{Examples}
Request performing \radfet experiment with 100 s biasing delay, 4 samples taken and save results
to '/radfet' file with communication correlation id set to 12.

\exampleCall{PerformRadFETExperiment(
    correlation_id=12,
    delay=10,
    samples_count=4,
    output_file_name='/radfet')}
