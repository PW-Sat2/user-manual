\telecommand{Perform RadFET Experiment}

\tccategory{PD}

\subsection{General description}
\texttt{PerformRadFETExperiment} telecomand can be used to request performing an \radfet 
experiment at first given opportunity.

\paragraph{Known issues} \mbox{} \\
\None

\paragraph{Side effects} \mbox{} \\
This command overrides previously requested experiments that has not been started yet. 

\paragraph{Usage limitations} \mbox{} \\
This command does not ensure that \radfet experiment will be started by the time that 
response is sent back to the operator. It only requests performing an experiment, which 
will be done at earliest opportunity.

\paragraph{Recommendations for operation team} \mbox{} \\
The experiment will overwrite selected output file if it already exists therefore ensure 
that selected file does not already exists or is no longer needed or some important data
may be lost.

\subsection{Definition}
\TelecommandApid{PerformRadFETExperiment}
\TelecommandDeclaration{PerformRadFETExperiment}

Parameters: 

\begin{tcarglist}
    \tcarg{correlation\_id}{8}{Correlation ID}
    \tcarg{delay}{8}{Time delay in seconds between subsequent \radfet measurements}
    \tcarg{samples\_count}{8}{Number of measurements to take}
    \tcarg{output\_file\_name}{up to 30 * 8}{Path to file that should contain experiments results}
\end{tcarglist}

\subsection{Responses}
\ResponseApid{ExperimentSuccessFrame}
\paragraph{ExperimentSuccessFrame}

Correlated response frame \texttt{ExperimentSuccessFrame} confirming 
that \radfet experiment performance has been requested.

\paragraph{ExperimentErrorFrame}
Correlated response frame \texttt{ExperimentErrorFrame} representing an error. 
Error code indicates failure reason:

\begin{tabular}{r | l}
    Error code & Value \\
    \hline
    \texttt{0x00}   & Success \\
    \texttt{0x01}   & Malformed (too short) telecommand \\
    \texttt{0x02}   & Another experiment may already be running \\
\end{tabular}

\subsection{Examples}
Request performing \radfet experiment with 4 samples taken once every second and save results
to '/radfet' file with communication correlation id set to 12.

\exampleCall{PerformRadFETExperiment(
    correlation_id=12,
    delay=1,
    samples_count=4,
    output_file_name='/radfet')}
