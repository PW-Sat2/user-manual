\telecommand{Get compile info}

\tccategory{S}

\subsection{General description}
\texttt{GetCompileInfoTelecommand} telecommand can be used to obtain detailed build and version information of the software currently running on board.

\paragraph{Known issues} \mbox{} \\
\None

\paragraph{Side effects} \mbox{} \\
\None

\paragraph{Usage limitations} \mbox{} \\
\None

\paragraph{Recommendations for operation team} \mbox{} \\
Always send this telecommand after uploading new software version or rebooting into different boot slots to verify that correct version of \obc software is in use.

\subsection{Definition}
\TelecommandApid{GetCompileInfoTelecommand}
\TelecommandDeclaration{GetCompileInfoTelecommand}

No parameters are required for this telecommand.

\subsection{Responses}
\responseFrame{CompileInfoFrame}
\ResponseApid{CompileInfoFrame}

This frame contains full or partial textual description of current \obc software version in the following format:\bigskip\\
\begin{verbatim}
<BUILD_TYPE> Build from '<GIT_COMMIT_HASH>' branch: <GIT_BRANCH>
MCU: <TARGET_MCU_PLATFORM>
Payload: <TARGET_PLD_PLATFORM>
Machine: <HOST_NAME>
Built in <BINARY_DIR>
GCC: <GCC_VERSION>
\end{verbatim}
Where:
\begin{itemize}
	\item \texttt{<BUILD_TYPE>} is the name of the current build. It can be either \texttt{Release} or \texttt{Debug}.
	\item \texttt{<GIT_BRANCH>} is the name of git branch from which currently running OBC software has been built.
	\item \texttt{<GIT_COMMIT_HASH>} is the git commit hash from which currently running OBC software has been built.
	\item \texttt{<TARGET_MCU_PLATFORM>} is the current microcontroller platform name. This can be one of the following: 
	\begin{itemize}
		\item \texttt{EngModel} - for microcontroller engineering model variant,
		\item \texttt{FlightModel} - for microcontroller flight variant,
		\item \texttt{DevBoard} - for evaluation board.
	\end{itemize}
	\item \texttt{<TARGET_PLD_PLATFORM>} is the current payload platform name. This can be one of the following: 
	\begin{itemize}
		\item \texttt{FlightModel} - for payload flight model variant,
		\item \texttt{DM} - for payload mock.
	\end{itemize}
	\item \texttt{<HOST_NAME>} - name of the host machine that produced this \obc software build.
	\item \texttt{<BINARY_DIR>} - path to the build directory on the current machine.
	\item \texttt{<GCC_VERSION>} - GCC compiler version used to build the \obc software.
\end{itemize}

In case single \texttt{CompileInfoFrame} will not be able to hold entire version information there the \obc will split the version information into parts and will send them in separate \texttt{CompileInfoFrame}s each containing subsequent parts of the information string. 

\subsection{Examples}
Get version info of currently running \obc software

\exampleCall{GetCompileInfoTelecommand()}

For current version of flight software it should return:
\begin{verbatim}
Release Build from '07208653' branch: master
MCU: FlightModel
Payload: FlightModel
Machine: FP-PC3266
Built in C:/Jenkins/jobs/OBC/branches/master/workspace/build
GCC:5.4.1 20160919 (release) [ARM/embedded-5-branch revision 240496] 
\end{verbatim}

