\telecommand{PurgePhotoTelecommand}

\tccategory{S}

\subsection{General description}
\texttt{PurgePhotoTelecommand} purges current photo queue and resets photo service to it's default state. Disabling LCLs is done in asynchronous manner so success response for this telecommand \textbf{DOES NOT} indicate correct power off of cameras.


\paragraph{Known issues}
\begin{itemize}
	\item If current photo processing has encountered serious error and is hanging purging photo queue \textbf{WILL NOT} solve that situation. Full power cycle is required.
	\item Success response does not indicate that CAM LCLs have been turned off. This can be verified in telemetry.
\end{itemize}

\paragraph{Side effects}
\begin{itemize}
	\item Abort of queued photo operations
\end{itemize}

\paragraph{Usage limitations}\mbox{}\\ \None

\paragraph{Recommendations for operation team}
\begin{itemize}
	\item Use this command to ensure empty photo processing pipeline before queueing another photo operation.
	\item Wait for response before queueing next operation
\end{itemize}


\subsection{Definition}
\TelecommandApid{PurgePhotoTelecommand}
\TelecommandDeclaration{PurgePhotoTelecommand}

Parameters: 

\begin{tcarglist}
	\tcarg{correlation\_id}{8}{Correlation ID}
\end{tcarglist}


\subsection{Responses}

\paragraph{PurgePhotoSuccessFrame}
\ResponseApid{PurgePhotoSuccessFrame}
Correlated response frame \texttt{PurgePhotoSuccessFrame} represents success. 

\paragraph{PurgePhotoErrorFrame}
\ResponseApid{PurgePhotoErrorFrame}
Correlated response frame \texttt{PurgePhotoErrorFrame} represents error. Error code indicates failure reason:

\begin{tabular}{r | l}
	Error code & Value \\
	\hline
	\texttt{1} & Malformed (too short) telecommand \\	
\end{tabular}


\subsection{Example usage}
Purge photo queue, correlation ID = 11
\exampleCall{PurgePhotoTelecommand(11)}

